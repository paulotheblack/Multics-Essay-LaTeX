
\section{Significance and Influence of Multics}

As we could read in previous sections, it is clear that Multics was the first system, which considered \textit{security} a 
\textbf{neccessity}, not an additional feature. Multics was designed to be secure from the begining. This fact is in modern 
operating systems taken for granted. In the xxxx the system was awarded the B2 security rating by the NCSC, the first 
(and for years only) system to get a B2 rating.

Multics was written in the \textit{PL/I} language (created by IBM in 1965), only part of the operating system was implemented 
in \textit{assembly} langugage. Writing operating system in a \textit{high-level} (not in today's standarts) language was a 
radical idea at the time. In addition to PL/I supports BCPL, BASIC, FORTRAN, LISP, COBOL, ALGOL68 and additionaly C and Pascal. 

Multics provided \textit{first commercial relational database} product. The \textit{MRDS}, Multics Relational Data Store.

All systems in early 70's was designed to be able to run 24/7, but only Multics had capability to add or remove CPUs, memory, 
I/O controllers and disk drives from system configuration while the system is running.

The most notable influence on other operating system, which we can still see today is in GNU/Linux .based systems.
Part of it because inventors of \textbf{UNIX} \textit{Ken Thompson} and \textit{Dennis Ritchie}(also inventor of C language), 
worked on Multics until \textit{Bell Labs} dropped out of the Multics development in 1969.
Ken Thompson was "inspired" by \textit{access control list mechanism}, which was implemented in UNIX and was also one of 
primary security measurements in the system.
The idea of having the \textit{command processing shell} to be an ordinary slave program came from the Multics design, 
and a predecessor program on \textit{CTSS} by \textit{Louis Pouzin} called \textbf{RUNCOM}.

The vaule of Multics in UNIX early development ideas could be also seen at the \textit{50th Anniversary} of the Unix creation.
One of the articles, by \textit{Richard Jensen} is called \textit{Unix at 50: How the OS that powered smartphones started from 
failure}, which direct reference to Multics. Such articles often say: \textit{"Multics failed. The Bell Labs computer guys then 
invented their own system, Unix, and Unix succeeded."}. On the other hand, Multics did not fail, it accomplished almost all of 
its stated goals. Unix succeeded too, solving a different problem.

Next time, when you "ssh-into" your remote server and see greeting such as this: \newline
\textit{Last login: Fri Apr 17 13:52:12 2020 from 88.212.40.17} \newline
You will think of \textbf{Multiplexed Information and Computing Service}.




\section{Glossary}
% System designer
% Access
% Administator
% Bottleneck
% Ring mechanism
% Descriptor
% Encryption device
% Reference Monitor
% Subverter
% Gatekeeper
% Gate segment
% WWMCCS GCOS
% IBM OS/360/370
% Tiger team
% IDC - Incerement address, Decrement tally, and Continue)
% ITS (pointer) -
% RCU - Restore Control Unit (ring 0 instruction)
% Signaller - Module responsivle for procesing faults (master mode procedure)
% PDS - Process Data Segment
% Penetrator - 
% PL/I - 
% B2 mark

% Bibliography
