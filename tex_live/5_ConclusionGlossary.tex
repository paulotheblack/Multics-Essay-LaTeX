
\section{Significance and Influence of Multics}

As we could read in previous sections, it is clear that Multics was the first system, which considered \textit{security} a 
\textbf{necessity}, not an additional feature. Multics was designed to be secure from the beginning. This fact is in modern 
operating systems taken for granted. In the 1985 the system was awarded the B2 security rating by the NCSC, the first 
(and for years only) system to get a B2 rating.

Multics was written in the \textit{PL/I} language (created by IBM in 1964), only part of the operating system was implemented 
in \textit{assembly} language. Writing operating system in a \textit{high-level} (not in today's standards) language was a 
radical idea at the time. In addition to PL/I, Multics supports BCPL, BASIC, FORTRAN, LISP, COBOL, ALGOL68 and additionally C and Pascal. 

Multics provided \textit{first commercial relational database} product. The \textit{MRDS}, Multics Relational Data Store.\cite{mindpride}

All systems in early 70's was designed to be able to run 24/7, but only Multics had capability to add or remove CPUs, memory, 
I/O controllers and disk drives from system configuration while the system is running.

The most notable influence on other operating system, which we can still see today is in GNU/Linux based systems.
Part of it is because inventors of \textbf{UNIX} \textit{Ken Thompson} and \textit{Dennis Ritchie} (also inventor of C language), 
worked on Multics until \textit{Bell Labs} dropped out of the Multics development in 1969.

Ken Thompson was "inspired" by \textit{access control list mechanism}, which was implemented in UNIX and was also one of the
primary security measurements in the system.

The idea of having the \textit{command processing shell} to be an ordinary slave program came from the Multics design, 
and a predecessor program on \textit{CTSS} by \textit{Louis Pouzin} called \textbf{RUNCOM}.\cite{multicians}

The value of Multics in UNIX early development ideas could be also seen at the \textit{50th Anniversary} of the Unix creation.
One of the articles, by \textit{Richard Jensen} is called \textit{Unix at 50: How the OS that powered smartphones started from 
failure}, which direct reference to Multics. Such articles often say: 
\begin{displayquote}
"Multics failed. The Bell Labs computer guys then invented their own system, Unix, and Unix succeeded.
\end{displayquote}
On the other hand, Multics did not fail, it accomplished almost all of its stated goals. Unix succeeded too, solving a different problem.
\newline \newline
Next time, when you "ssh-into" your remote server and see greeting such as this:
\begin{lstlisting}
    Last login: Fri Apr 17 13:52:12 2020 from 88.212.40.17}
\end{lstlisting}
You can think of \textbf{Multiplexed Information and Computing Service}.


\begin{displayquote}
\textit{
"To what extent should one trust a statement that a program is free of Trojan horses? 
Perhaps it is more important to trust the people who wrote the software."
}
\textbf{Ken Thompson} \cite{thompson}
\end{displayquote}

\section{Glossary}
\begin{itemize}
    \item CTSS - The Compatible Time-Sharing System, \href{https://www.youtube.com/watch?v=Q07PhW5sCEk}{IBM 7090 and young prof. Corbato about Timesharing on WGBH-TV (1963)}
    \item Descriptor - Contents of an SDW.
    \item Gate - Segment that allows transfer of control between rings in a controlled fashion. Each gate segment has a vector of entries at its start.
    \item IBM OS\textbackslash 360\textbackslash 370 - Family of mainframe computer system developed by IBM.
    \item IDC - Incerement address, Decrement tally, and Continue
    \item PDS - Process Data Segment
    \item PL\textbackslash I - Programming Language \#1, invented by George Radin of IBM in 1964
    \item RCU - Restore Control Unit (ring 0 instruction)
    \item Signaller - Module responsible for processing faults (master mode procedure)
    \item SDW - Segment Descriptor Word. An element of a process's descriptor segment; the hardware-accessible data element that defines a segment and the process's access rights to it.
    \item Subverter - Program written by the Project ZARF team.
    \item WWMCCS GCOS -  The Worldwide Military Command and Control System General Comprehensive Operating Supervisor
    \item ZARF - Code name for Air Force/MITRE tiger team project that cracked Multics security in 1973.
\end{itemize}

