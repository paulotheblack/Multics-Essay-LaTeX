% Based on:
% [1] Multics Security Evaluation: Vulnerability analysis II
% Paul A. Karger (2nd Lt, USAF)
% Roger R. Schell (Major, USAF)

% [2] A brief description of privacy measures in the Multics operating system
% Edward L. Glaser (MIT, Project MAC)

% [3] Protection and control of information sharing in Multics
% Jerome H. Saltzer (MIT, Project MAC)


%# TODO Basic questions:
% Why:

% What:

% How:

% Conclusion:

%# TODO Glossary:
% System designer
% Access
% Administator
% Bottleneck

% # --------------------------------------------------------------------------------- #

\section{Design Principles}

\begin{enumerate}
    \item Every designer should know and understand the protection objectives of the system.
     There are many points at which \textit{"I don't care"} decision can be a breaking point 
     of internal security measurements. By keeping all desifners aware of the protection objectives, 
     the decisions are more likely to be made correctly.
     
     \item Keep the design as \textit{simple} and as \textit{small} as possible. Design and 
     implementation errors which result in unwanted acces paths will not be immediately noticed during 
     routine use. Therefore, techniques such as complete, \textit{line-by-line} auditing of the 
     protection mechanisms are necessary, thats the reason why small and simple design is 
     essential.

     \item Protection mechanism should be based on \textit{permission} rather than exclusion. 
     The principle means that the default situation is \textit{lack of access}, and the protection 
     scheme provides selective permission for specific purpose.

     \item Every access to every object must be checked for authority.
     
     \item The design is not secret. The mechanism do not depend on the \textit{ignorance of potential 
     attackers}, but rather on possession of specific, more easily protected, protection keys or 
     passwords. 

     \item The principle of least privilege. Every program and every privileged user of the system 
     should operate using the least amount of privilege necessary to complete the job.

     \item Make sure that the design encourages correct beahviour in the user, operators and 
     administators of the system. It is essential that human interface be designed for naturalness, 
     ease of use and simplicity, so that user will routinely and automatically \textit{apply the 
     protection mechanisms}.

     \item Provide the option of complete decentralization of the administation of protection 
     specifications. If the system design forces all administrative decisions to be set by a single 
     administator, that administrator quickly becomes a \textit{bottleneck}. which could result in users 
     begin adopting habits which bypass the administartor.

     \item The system provides a complete set of tools, so that a user without exercising special 
     privileges, may construct a \textit{protected subsystem}, which is a collection of programs 
     and data woth the property that the data may be accessed only by programs in the subsystem, and 
     the programs may be entered only at designated entry points.
\end{enumerate}


\section{Security Controls}

\subsection{Access Control List}






















\textbf{The basic security controls of Multics fall into three areas:}

\subsection{Hardware Security Controls}

\subsubsection{Segmentation Hardware}

Also viewed as the most \textit{fundamental} security controls in the HIS 645.
% # TODO segmentation && master mode dopisat

\subsection{Software Security Controls}

\subsubsection{Protection Rings} 
% # TODO spomenut, ze uz len par-krat v historii boli implementovane
% #? vcelku unikat

The primary software security control on 645 Multics system had been called \textit{The ring mechanism}.
Originally had been proposed as extension for the more traditional \textit{master and slave mode}.
The protection rigns had been numbered from 0 to 7, and definied in the way:

\textit{The higher numbered rings having less privilege than lower numbered rings, and ring 0
containing 'hardcore' supervisor}. On the 645 CPU rings mechanism have not been implemented in the way of hardware.
(Only after several years on HIS 6180 CPU.) Therefore, the protection rings have been implemented by providing 
\textit{eight descriptor segments} for each user, one descriptior segment per ring.

% # TODO protection rings doplnit info z inej literatury


\subsubsection{Access Control Lists}
% # TODO priame prebratie UNIX-om (kopirant Thompson)

Segments in Multics are stored in a hierarchy of directories. \textit{Directory} is a special type of of
segment that is not directly accessible to the(all) users.
Directory stores names and other information about undelying segments and directories.
Each segment and directory has an \textit{Access control list} (ACL) in its parent directory entry
\textit{who} may \textit{read}, \textit{write} or \textit{execute} (also known as 'r-w-x') the segment or obtain 
\textit{status} of, \textit{modify} entries line or \textit{append} entries on directory (also known as 's-m-a').

In order to acces file/directory, the ring 0 software compares the ID of user with access control list entries.
(User indentification is established when user sings on to the Multics and it os stored in the Process Data 
Segment (PDS), whish is accessible only in ring 0 or in Master mode - \textit{so that user cannot modify the
process data segment} and grant acces anywhere on the system.)

\subsubsection{Master Mode}
% # TODO prirovnat k superuser

The convention around protecting \textit{master mode software}, was specified as the master mode procedures
were not to be used outside ring 0. (Becauase of endless and penetrable checking of input arguments).
Convention had been definied as of "\textit{each master mode procedure must have a master mode pseudo-operation
 code assembled into location 0}".

 The master mode pseudo-operation generates code to test an index register for a value corresponding to an 
 entry point in the segment. If the index register is invalid, the master mode pseudo-operation saves the
 registers for debugging and shuts the system down.

 
 \subsection{Procedural Security Controls}

 \subsubsection{Enciphered passwords}
% # TODO neskor porovnaj, ze v tejto dobe je to rovnake (len vyuzite nove algols) no ludia to beztak zanedbavaju

As a primary authentication method (since the brink of computers until these days) during login sequence 
is password.

When user types a password it is \textit{enciphered} and compared with the stored encpithered version for validity.
All passwords on systetem are stored in non-invertible cipher form, which also acts as additional security in case 
of system dump. Plain text passwords are stored nowhere on the system.

\subsubsection{Login Audit}
% # TODO prirovnat -> prihlasenie na server totozne hlasenie

Each login and logout is carefully audited to check for attempts to guess a valid password for a certain user.
In addition each user is informed of the \textit{date}, \textit{time} and \textit{terminal identification} of last
login to detect past compromises of the users \textit{access rights}.
Also as addition, the user is told the number of times his password has been given incorrectly since its 
last correct use. 

\subsubsection{Software Maintenancee Procedures}
% # TODO prirovnat k sucasnemu fungovaniu package managers

The maintenance of the \textit{Multics software} had been carried out textbf{online} on a \textit{dial-up 
Multics facility}. A programmer prepared (and debugs) his software for installation, then submits his
software to a \textit{library installer} who copied and recompiled the source code in a \textit{protected 
directory}. 
The software prior installing into the system source and object libraries had been checked by library installer.

Ring 0 software is stored on a \textit{system tape} that is reloaded into the system each time it is brought up.
(New system tapes are generated from online copies of the ring 0 software).

The system libraries had been protected against modification by access control list mechanism. In addition 
the library installers had been periodically checked of last modification of all segments in the library to
detect unathorized modifications.