\section{Running Multics nowadays}

Creating a Multics system or machine nowadays is a bit sketchy process. Since Multics is reling on designated hardware 
such as HIS-645 or 6180 processors (developed by Honeywell) it is not an easy task to do.
You can not simply create \textit{Virtual Machine} from from disk image (ISO). But there are still some options how to do it, 
you can use \textit{simulator/emulator} for such a task. 

MIT in cooperation with BULL runs \href{https://web.mit.edu/multics-history/source/Multics_Internet_Server/Multics_sources.html}{Multics Internet Server}
containing edition of Multics software and documentation, which is useful if you want to try PL/I and rest locally.


\subsection{DPS8-Simulator}

I came accross project which still maintain love and desire for running Multics system.
\textbf{RingZero - Multics Reborn} project, created by \textit{Harry Reed} and \textit{Charles Anthony} which is available for 
public \href{http://ringzero.wikidot.com/}{download}. Development takes place on \href{https://gitlab.com/dps8m/dps8m}{GitLab}. 
They provide \textit{DPS8-M simulator}. Still with no much capabilities, but as stated milestone from 11. August 2014:
\textit{Multics reborn! The dps-8/m SIMH-based simulator booted Multics MR 12.5, came to operator command level, entered 
admin mode, created a small PL/I program, compiled and executed it, and shut down.} PRIDAJ REFERENCIU[ring zeo]


SEM DAJ NAVOD AKO TO ROZBEHAT U SEBA

stiahnut dps8m
stiahnut quickstart 
setup http://multics-wiki.swenson.org/index.php/Getting_Started
using http://multics-wiki.swenson.org/index.php/Using_Multics#More_Processing
%# TODO pridaj do ref.

