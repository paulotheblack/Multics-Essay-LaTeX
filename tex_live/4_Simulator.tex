% Based on:
% setup http://multics-wiki.swenson.org/index.php/Getting_Started
% using http://multics-wiki.swenson.org/index.php/Using_Multics#More_Processing
%
% # --------------------------------------------------------------------------------- #

\section{Running Multics nowadays}

Creating a Multics system or machine nowadays is a bit sketchy process. Since Multics is reling on designated hardware 
such as HIS-645 or 6180 processors (developed by Honeywell) it is not an easy task to do.
You can not simply create \textit{Virtual Machine} from from disk image (ISO). But there are still some options how to do it, 
you can use \textit{simulator/emulator} for such a task. 

MIT in cooperation with BULL runs \href{https://web.mit.edu/multics-history/source/Multics_Internet_Server/Multics_sources.html}{Multics Internet Server}
containing edition of Multics software and documentation, which is useful if you want to try PL/I and rest locally.


\subsection{DPS8-Simulator}

I came accross project which still maintain love and desire for running Multics system.
\textbf{RingZero - Multics Reborn} project, created by \textit{Harry Reed} and \textit{Charles Anthony} which is available for 
public \href{http://ringzero.wikidot.com/}{download}. Development takes place on \href{https://gitlab.com/dps8m/dps8m}{GitLab}. 
They provide \textit{DPS8-M simulator}. Still with no much capabilities, but as stated milestone from 11. August 2014:

%# TODO qoute \usepackage{csquotes}
\begin{displayquote}
Multics reborn! The dps-8/m SIMH-based simulator booted Multics MR 12.5, came to operator command level, entered 
admin mode, created a small PL/I program, compiled and executed it, and shut down.
\end{displayquote}

\subsection{How to run Multics on GNU\Linux}
First of all we need to \href{ringzero.wikidot.com/local--files/start/dps8m_linux.tgz}{download} 
\textit{Pre-built executable} (so we do not need to build simulator from source).

\href{https://s3.amazonaws.com/eswenson-multics/public/releases/MR12.6f/QuickStart_MR12.6f.zip}{Download} "QuickStar RLV", 
which provides and already Multics system disk image that is ready to run.


Unzip both archives and put in the same directory in system. Go to direcotry in terminal and:
\begin{lstlisting}
$> ./dps8 MR12.6f_boot.ini 

DPS8/M emulator (git 2a56f38d)
Production build
#### M_SHARED BUILD ####
System state restored
Please register your system at https://ringzero.wikidot.com/wiki:register
or create the file 'serial.txt' containing the line 'sn: 0'.
FNP telnet server port set to 6180

DPS8/M simulator V4.0-0 Beta        git commit id: c420925a
TAPE: unit is read only
MR12.6f_boot.ini-30> set opcon config=attn_hack=1
Non-existent device
[FNP emulation: listening to 127.0.0.1 6180]
CONSOLE: ALERT
bootload_0: Booting system MR12.6f generated 01/09/17 1119.1 pst Mon.   
1500.2  announce_chwm: 428. pages used of 512. in wired environment.
1500.2  announce_chwm: 706. words used of 1024. in int_unpaged_page_tables.
find_rpv_subsystem: Enter RPV data: M-> [auto-input] rpv a11 ipc 3381 0a

1500.2  load_mst: 946. out of 1048. pages used in disk mst area.
bce (early) 1500.2: M-> [auto-input] bce

Multics Y2K.  System was last shudown/ESD at:
Saturday, April 1, 2017 20:37:12 pst
Current system time is: Sunday, April 19, 2020 7:00:16 pst.
Is this correct? M-> [auto-input] yes

The current time is more than the supplied boot_delta hours beyond the
unmounted time recorded in the RPV label.  Is this correct? M-> [auto-input] yes

The current time I'm using is more than 12 hours
after the last shutdown time recorded in the RPV label.
Are you sure this is correct? M-> [auto-input] yes

bce (boot) 0700.2: M-> [auto-input] boot star

Multics MR12.6f - 04/19/20  0700.4 pst Sun
0700.4  Loading FNP d, >user_dir_dir>SysAdmin>a>mcs.7.6c>site_mcs 7.6c
Received BOOTLOAD command...
0700.4  FNP d loaded successfully


Volumes to be Scavenged:

rpv    (root)
root2  (root)
root3  (root)

Ready
0700  as   as_init_: Multics MR12.6f; Answering Service 17.0
0700  as   LOGIN              IO.SysDaemon dmn cord (create)
0700  as   LOGIN              Backup.SysDaemon dmn bk (create)
0700  as   LOGIN              IO.SysDaemon dmn prta (create)
0700  as   LOGIN              Utility.SysDaemon dmn ut (create)
0700  as   LOGIN              Volume_Dumper.Daemon dmn vinc (create)
0700  as   LOGIN              Scavenger.SysDaemon dmn scav1 (create)
0700  as   as_mcs_mpx_: Load signalled for FNP d.
0700  bk   
0700  cord Enter command:  coordinator, driver, or logout:
-->  cord
0700  bk   r 07:00 0.194 30
0700  bk   
-->  bk
0700  prta Enter command:  coordinator, driver, or logout:
-->  prta
0700  ut   copy_dump: Attempt to re-copy an invalid dump.
0700  ut   delete_old_pdds: Some directory or segment in the pathname is not listed in the VTOC. Unable to salvage >pdd.!BBBKPBBzF
0700  ut   jWCfq. Will attempt to delete it.
0700  ut   delete_old_pdds: Some directory or segment in the pathname is not listed in the VTOC. Unable to salvage >sl1.!BBBKPBBzF
0700  ut   jWCfq. Will attempt to delete it.
0700  vinc 
0700  vinc r 07:00 0.197 25
0700  vinc 
-->  vinc
0700  scav1 Created >system_control_1>scav1.message
0700.5  scavenger: Begin scavenge of dska_00a by Scavenger.SysDaemon.z
0700  scav1 Scavenging volume rpv of logical volume root
0700  as   sc_admin_command_: Utility.SysDaemon.z: delete_old_pdds
0700  ut   send_admin_command: Execution started ... 
0700.7  vtoce_stock_man: Attempt to deposit free vtocx 4037 on dska_00
0700.7  vtoce_stock_man: Attempt to deposit free vtocx 4036 on dska_00
0700  ut   completed.
0700.7  scavenger: Scavenge of dska_00a by Scavenger.SysDaemon.z completed.
0700.7  scavenger: Begin scavenge of dska_00b by Scavenger.SysDaemon.z
0700  scav1 Scavenging volume root2 of logical volume root
0700  ut   monitor_quota: The requested action was not performed. 
0700  ut   The quota of >dumps is 0, a record limit needs to be specified.
0700.7  RCP: Attached tapa_00 for Utility.SysDaemon.z
0700  ut   
0700  ut   Records   Left  %    VTOCEs   Left  %   PB/PD  LV Name
0700  ut   
0700  ut   166172   99180  60   42218   33972  80  pb     root
0700  ut   
0700  ut   r 07:00 0.505 477
0700  ut   
-->  ut
0700.7  RCP: Detached tapa_00 from Utility.SysDaemon.z
0700.7  RCP: Attached rdra for Utility.SysDaemon.z
0700.7  RCP: Detached rdra from Utility.SysDaemon.z
0700.7  RCP: Attached puna for Utility.SysDaemon.z
0700.7  RCP: Detached puna from Utility.SysDaemon.z
0700.7  RCP: Attached prta for Utility.SysDaemon.z
0700.7  RCP: Detached prta from Utility.SysDaemon.z
0700.9  scavenger: Scavenge of dska_00b by Scavenger.SysDaemon.z completed.
0700  scav1 Scavenging volume root3 of logical volume root
0700.9  scavenger: Begin scavenge of dska_00c by Scavenger.SysDaemon.z
0701.1  scavenger: Scavenge of dska_00c by Scavenger.SysDaemon.z completed.
\end{lstlisting}

Your Multics system is up and running. The terminal shell in which you started the simulator and Multics is now the operator console.
The simulator should be listing on TCP port 6180 and if you connect to that port using telnet, you should be able to login. 

\begin{lstlisting}
$> telnet localhost 6180

Trying ::1...
telnet: connect to address ::1: Connection refused
Trying 127.0.0.1...
Connected to localhost.
Escape character is '^]'.
HSLA Port (d.h000,d.h001,d.h002,d.h003,d.h004,d.h005,d.h006,d.h007,d.h008,d.h009,d.h010,d.h011,d.h012,d.h013,d.h014,d.h015,d.h016,d.h017,d.h018,d.h019,d.h020,d.h021,d.h022,d.h023,d.h024,d.h025,d.h026,d.h027,d.h028,d.h029,d.h030,d.h031)? 
Attached to line d.h000

Multics MR12.6f: Installation and location (Channel d.h000)
Load = 5.0 out of 90.0 units: users = 5, 04/19/20  0706.0 pst Sun
\end{lstlisting}

Now you can login as SysAdmin

\begin{lstlisting}
$> login Repair -cpw    # flag to change password

Password:
New Password:
New Password Again:

Your password was given incorrectly at 04/15/20  0706.4 pst Sun from ASCII term
\cinal "none".      # Audit Trial in use, I've put password once incorrectly

Password changed.
You are protected from preemption.
Repair.SysAdmin logged in 04/19/20  0706.7 pst Sun from ASCII terminal "none".

New messages in message_of_the_day:

Welcome to the Multics System.

print_motd:  Created >user_dir_dir>SysAdmin>Repair>Repair.value.
r 07:06 0.208 31

M->         # you can start using Multics
\end{lstlisting}

With this running Multics you can do:

\begin{itemize}
    \item Logging into an Existing Account
    \item Logging in as an Anonymous User
    \item Creating Directories
    \item Listing the Contents of Directories
    \item Entry Names
    \item Editing files (Emas, qedx, edm, Teco)
    \item Printing files
    \item Setting Time Zone
\end{itemize}

\textbf{Each of this I will show in presentation}
% SEM DAJ NAVOD AKO TO ROZBEHAT U SEBA

% stiahnut dps8m
% stiahnut quickstart 
% setup http://multics-wiki.swenson.org/index.php/Getting_Started
% using http://multics-wiki.swenson.org/index.php/Using_Multics#More_Processing
% %# TODO pridaj do ref.


